\documentclass[11pt]{article}
%\usepackage{geometry}                % See geometry.pdf to learn the layout options. There are lots.
%\geometry{letterpaper}                   % ... or a4paper or a5paper or ... 
%\geometry{landscape}                % Activate for for rotated page geometry
%\usepackage[parfill]{parskip}    % Activate to begin paragraphs with an empty line rather than an indent
%\usepackage{graphicx}
%\usepackage{amssymb}
%\usepackage{epstopdf}
%\DeclareGraphicsRule{.tif}{png}{.png}{`convert #1 `dirname #1`/`basename #1 .tif`.png}

\usepackage{fancyhdr}
\usepackage{lastpage}

\pagestyle{fancy}
\fancyfoot[L]{Project X \\ CST 200}
\fancyfoot[C]{Proposal Worksheet \\ {\copyright} Preston Lee 2010. All rights reserved.}
\fancyfoot[RO, LE] {{\thepage} of \pageref{LastPage}}

\title{Project X: Proposal Worksheet}

\author{Preston Lee}
\date{}                                           % Activate to display a given date or no date

\begin{document}
\maketitle
\tableofcontents

\section{Abstract}
Project X is a self-defined team project that uses the combined brain power of your team to define, design, build, test and demo a meaningful Java programming project. This worksheet will take you through the initial steps team and project planning steps. {\bf This a proposal that must be ``approved" by the instructor prior to development.}

\pagebreak
 
\section{Build the team!}
Self assemble yourselfs into teams of your choosing. Team requirements:
\begin{enumerate}
\item Three or four people. {\it No exceptions}.
\item At least one person with {\it no} previous Java courses.
\item At least one person with previous Java courses.
\item Captain must have no previous Java courses.
\end{enumerate}

\subsection{Pick some cool names.}
Every project team needs some cool names, as well as a captain. Fill in the blanks:
%\vspace{.2in}

\hspace{.5in}{\bf Team Captain: }

\hspace{.5in}{\bf Team Code Name: }

\hspace{.5in}{\bf Project Code Name: }

%\vspace{.2in}

\subsection{Gather member information.}
Fill in the following table with your team information
\begin{table}[htdp]
\caption{Team member information.}
\begin{center}
\begin{tabular}{c c c c c}
\hline\hline
Number & Name & ASURite & Previous Course & Email \\ [.5ex]
\hline
(e.g.) & Alice Anderson & aanderson & CST 100 (Java) & aanderson \\
\hline
{\bf \#1} & & & & \\ [1ex]
{\bf \#2} & & & & \\ [1ex]
{\bf \#3} & & & & \\ [1ex]
{\bf \#4} & & & & \\ [1ex]
\hline
\end{tabular}
\end{center}
\label{table:team}
\end{table}

\pagebreak

\section{BYO project.}

\subsection{Define the problem.}

Define a one-paragraph problem statement that you will attempt to solve. (Tip: Pick a {\it practical} problem that you think you and your team can realistically solve in about a months time.)


\underline{\phantom{\hspace{5in}}}

\underline{\phantom{\hspace{5in}}}

\underline{\phantom{\hspace{5in}}}

\underline{\phantom{\hspace{5in}}}

\underline{\phantom{\hspace{5in}}}

\underline{\phantom{\hspace{5in}}}

\subsection{Demonstrate your knowledge.}

For this proposal to be approved, your project {\bf must} prove your knowledge of course material by demonstrating {\it all} of the following:

\begin{description}
\item[Timeliness] Real-world projects can't wait until the last minute, and neither should you. The development of your project should be paced throughout the time you are given, {\bf not} crammed into the last week.
\item[Abstraction] Your must ``reuse" code (in a meaningful way) by using inheritance and well as polymorphic references where sensical in your application.
\item[Input] The application must accept input from the user, disk, network or other external source.
\item[Output] Programs must provide meaningful real-time output to the use. It's ok to write to disk, but you still must print to the screen or put up some sort of GUI. ({\bf Bonus}: Implement a GUI using Swing or SWT.)
\item[Exception Handling] Any/All user, disk, network etc. I/O must be ``solid": written with thorough exception handling practices to prevent provide a reasonable level of application robustness. Use try/catch and ({\bf Bonus}: Make use of a network connection.)
\item[Code Documentation] Code with documentation on usage and programmer thinking is a magnitude more valuable than code without. All code must be thoroughly comment in JavaDoc format. ({\bf Bonus}: Provide ``howto" documentation on all reusable classes.)
\item[Automated Test Cases] Provide a suite of "unit tests" that allow you to quickly run regression tests against your code base. You must also include ``negative'' test cases: code which intentionally calls functions with invalid input to verify that the code fails the way it is supposed to. For example, passing null, invalid numbers etc. to functions expecting ``correct" input should do something reasonable. ({\bf Bonus}: Use a unit test framework.) 
\end{description}

\subsection{Project summary.}
In 2-3 paragraphs, summarize the purpose, input, behavior, and expected output of your application.

\underline{\phantom{\hspace{5in}}}

\underline{\phantom{\hspace{5in}}}

\underline{\phantom{\hspace{5in}}}

\underline{\phantom{\hspace{5in}}}

\underline{\phantom{\hspace{5in}}}

\underline{\phantom{\hspace{5in}}}

\underline{\phantom{\hspace{5in}}}

\underline{\phantom{\hspace{5in}}}

\underline{\phantom{\hspace{5in}}}

\underline{\phantom{\hspace{5in}}}

\underline{\phantom{\hspace{5in}}}

\underline{\phantom{\hspace{5in}}}

\underline{\phantom{\hspace{5in}}}

\underline{\phantom{\hspace{5in}}}

\underline{\phantom{\hspace{5in}}}

\underline{\phantom{\hspace{5in}}}

\underline{\phantom{\hspace{5in}}}

\underline{\phantom{\hspace{5in}}}

\underline{\phantom{\hspace{5in}}}

\underline{\phantom{\hspace{5in}}}

\underline{\phantom{\hspace{5in}}}

\underline{\phantom{\hspace{5in}}}

\underline{\phantom{\hspace{5in}}}

\underline{\phantom{\hspace{5in}}}

\subsection{Target user(s).}
In 1-2 paragraphs, describe the intended user of your application. (If you're writing a game, for example, is it for kids or adults? ...fun or educational purposes? ...paying customers or free access?)


\underline{\phantom{\hspace{5in}}}

\underline{\phantom{\hspace{5in}}}

\underline{\phantom{\hspace{5in}}}

\underline{\phantom{\hspace{5in}}}

\underline{\phantom{\hspace{5in}}}

\underline{\phantom{\hspace{5in}}}

\underline{\phantom{\hspace{5in}}}

\underline{\phantom{\hspace{5in}}}

\underline{\phantom{\hspace{5in}}}

\subsection{Milestones.}
The team will need to hit all of the following milestones for full credit.

\begin{description}
\item[\bf Thursday, September 16th] Proposals due. 
\item[\bf Thursday, September 23th] Last chance acceptance date. 
\item[\bf Thursday, October 7th] Sanity check. Your team (led by the captain) will walk me through your progress thus far. By this point I expect you to have a good start on the project objectives as well as a good working relationship (and process) with the team.
\item[\bf Thursday, October 28th] Demo and presentation. Your team--lead by the captain--will deliver a well rehearsed project presentation and application demonstration in front of the class.
\end{description}


\pagebreak

\section{Define success.}
All team members will receive the same grade for the project. Additionally, the team captain will deliver a live 10-15 minute demo in front of the class prior to grading.

\subsection{Grading breakdown.}

Total points possible: 36. (Tip: This is a lot, so take the project seriously.)

\begin{description}
\item[20\%] In-class demo. {\it Be prepared.} Be professional. Quality of content, delivery and project demo will be considered. In other words, make sure your project doesn't ``blow up" in front of the class! You will have projector access to show slides, codes, and other additional resources you deem fit. 
\item[40\%] Given by instructor, based on the previously defined criteria.
\item[40\%] Defined by your team, below.
\end{description}

\subsection{Custom grading criteria.}

Define at least {\bf four} specific ways your project should be graded. (For example: {\it ``Correctly retrieves automated Google search results"}, or {\it ``Renders MineSweeper board correctly."}) These must be reasonably challenging criteria for a team of 3-4 people given over a month of development time. Trivial criteria such as {\it ``Application accepts invalid user input without crashing."} will be rejected.

\underline{\phantom{\hspace{5in}}}

\underline{\phantom{\hspace{5in}}}

\underline{\phantom{\hspace{5in}}}

\underline{\phantom{\hspace{5in}}}

\underline{\phantom{\hspace{5in}}}

\underline{\phantom{\hspace{5in}}}

\underline{\phantom{\hspace{5in}}}

\underline{\phantom{\hspace{5in}}}

\underline{\phantom{\hspace{5in}}}

\underline{\phantom{\hspace{5in}}}

\underline{\phantom{\hspace{5in}}}

\underline{\phantom{\hspace{5in}}}

\underline{\phantom{\hspace{5in}}}



\end{document}  